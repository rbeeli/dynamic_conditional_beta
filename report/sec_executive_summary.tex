\section*{Executive Summary}

\begin{doublespacing}

The beta measure was introduced by the renowned Capital Asset Pricing Model (CAPM) of \citeA{WS:64} and \citeA{JL:65}. It is a cornerstone in asset pricing theory and still widely used in practice, despite earning a lot of critique because of its poor empirical performance. The measure is the sensitivity of the security returns to changes in the market portfolio returns. The economic intuition behind the CAPM is attractive and easily understood, but \citeA{FF:04} showcase poor empirical performance of the model compared to the expectations of the theoretical framework.

More advanced estimation methods which account for heteroscedasticity, volatility-clustering and time-dependency exist, e.g. GARCH-based models. \citeA{BET:16} estimate stock betas using a time-varying conditional correlation model in conjunction with a GARCH model, also called DCC-GARCH, which was introduced by \citeA{EN:02}.

\citeA{BET:16} find a significant positive relation between conditional betas and the cross section of daily stock returns using the DCC-GARCH model for beta estimation. They estimate betas for all stocks in their sample for each day using a moving window. Based on the estimated betas, the cross-section of stock returns is predicted using beta deciles. They show that stocks in the lowest beta decile have significantly lower returns on the next trading day compared to stocks in the highest beta decile. The difference portfolio of \citeA{BET:16}, which takes a long position in stocks in the highest beta decile and a short position in stocks in the lowest beta decile, produces average returns and alphas in the range of 0.60\%–0.80\% per month. The strategy was backtested using all U.S.-based common stocks trading on the NYSE, AMEX, and NASDAQ exchanges with a stock price of \$5/share or more and with a market capitalization greater than \$10 million. The sample period starts in July 1963 and ends in December 2013.

The replication of this finding is the first objective of this thesis. We are able to replicate their findings with our sample data based on the DCC-GARCH construct and show that the value-weighted High-Low beta portfolio yields even higher monthly excess returns and alphas in the range of 1.7\%–1.9\%. After incorporating the estimated transaction costs of 35 basis points, the strategy yields monthly excess returns and alphas in the range of 1.35\%–1.55\%, which is almost twice the monthly excess return \citeA{BET:16} achieved in their study. We use a more recent time period and a different sample to test the strategy. Our sample covers all \indexName{} constituents and covers the period from \periodFrom{} to \periodTo{}. Additionally, we used an exponential moving average on the covariance estimates, which increased performance and lowered the portfolio turnover rate, allowing the strategy to be implemented at lower costs.

The second objective of this thesis is to replace the DCC-GARCH construct used by \citeA{BET:16} with the so-called COMFORT model developed by \citeA{PAP:15}. The more realistic, statistically advanced COMFORT model allows to model all major stylized facts of financial returns, including volatility clustering, dynamics in the dependency structure, asymmetry, and heavy tails. It supports various extensions which account for time-varying correlation dynamics and a hybrid GARCH-SV extension for modeling shocks across assets, which are an additional source of dynamics in the correlations. The COMFORT-DCC model used throughout this thesis uses a fat-tailed, multivariate asymmetric Laplace distribution (MALap) in conjunction with the Dynamic Conditional Correlation model of \citeA{ES:01} for the correlation dynamics; the hybrid GARCH-SV extension is not employed. The superior capabilities of the COMFORT-DCC model should enable it to outperform the DCC-GARCH model in terms of more significant beta estimates and a significantly better performance of the investment strategy, but the empirical results show that neither does the performance increase, nor do we find lower portfolio risk.

There are various reasons why the COMFORT-DCC model does not outperform the DCC-GARCH construct in our setting. Covariance estimation errors impair our High-Low portfolio analysis, because reliable and consistent beta estimates are crucial for our study---any extreme values due to estimation errors directly bias our results. Our analysis shows that especially the highest beta values are prone to estimation errors and therefore worsen the performance of the High-Low portfolio. Also, the COMFORT-DCC model was estimated each day with a window size of 1000 days, which is approximately four times bigger than the window size used by \citeA{BET:16}. The DCC-GARCH model with a window size of 1000 days performs still better than the COMFORT-DCC model, leading to the conclusion that the window size alone is not the deciding factor for the worse performance. Furthermore, the bivariate estimation method introduces an additional degree of freedom compared to a full-sample estimation approach for the mixing factor, because the model assumes that the mixing factor estimates used in COMFORT-DCC are the same for all asset returns. This additional degree of freedom likely introduces less precise/more random COMFORT-DCC model estimates.

The replication findings show promising results regarding the investment strategy used by \citeA{BET:16}, it outperforms the market and has higher risk-adjusted returns. Further studies of the COMFORT-DCC model with more reliable model estimates, different model extensions, and different estimation approaches are required in order to outperform the DCC-GARCH construct in our setting. In case of success, the resulting investment strategy would be highly interesting to investors.
 

\end{doublespacing}
