Empirical evidence shows that market betas vary substantially over time, hence time-varying beta models are of interest in the field of financial modeling. This thesis reexamines the findings of Bali et al. (2017) of a positive link between the dynamic conditional beta and the cross section of daily stock returns. Their investment strategy takes a long position in stocks in the highest beta decile and a short position in stocks in the lowest beta decile, and produces average returns and alphas in the range of 0.60\%–0.80\% per month. We are able to replicate their findings based on the DCC-GARCH construct and show that the value-weighted High-Low difference portfolio yields even higher monthly excess returns and alphas in the range of 1.7\%–1.9\% on our sample data. Replacing DCC-GARCH with the so-called COMFORT model, which is statistically more advanced and accounts for major stylized facts of financial asset returns, does not increase performance, nor does it result in lower portfolio risk due to model estimation errors and a longer estimation window.